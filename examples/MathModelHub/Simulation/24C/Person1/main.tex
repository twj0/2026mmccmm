\documentclass{mcmthesis}
\mcmsetup{CTeX = false,
        tcn = 2613942, problem = C,
        sheet = true, titleinsheet = true, keywordsinsheet = true,
        titlepage = false, abstract = true}
\usepackage{palatino}
\usepackage{amsmath}
\usepackage{amssymb}
\usepackage{graphicx}
\usepackage{subfigure}
\usepackage{booktabs}
\usepackage{multirow}
\usepackage{indentfirst}
\usepackage{float}
\usepackage{threeparttable}  % 实现表格+底部注释
\usepackage{tabularx}       % 自适应页面宽度的表格,避免内容超出页边距
\title{Capturing the Ebb and Flow: A Dynamic Momentum Model for Tennis Match Analysis}

\begin{document}

\begin{abstract}
% Summary Sheet LaTeX Code
\setlength{\parindent}{1.5em}

Tennis momentum swings are frequently decisive yet remain hard to quantify. This paper develops a comprehensive \textbf{Dynamic Momentum Score (DMS)} framework to identify, validate, and predict match flow. Utilizing point-level data from the \textbf{2023 Wimbledon Championships (7,284 points)}, we established a recursive scoring system based on \textbf{ELO-rating principles}. The model integrates a serve-advantage calibration (weighted by the observed \textbf{67.31\%} win rate), key-point multipliers, and a time-decay mechanism. \textbf{Results indicate that the average DMS achieves a 90.3\% accuracy in predicting match winners.}

To verify momentum as a physical reality, we conducted a \textbf{statistical validation suite}, including \textbf{Runs Tests} and \textbf{Conditional Probability Analysis}. The data-driven analysis reveals that a player's winning probability after a successful point rises to \textbf{54.4\%}, significantly exceeding the \textbf{51.0\%} baseline win rate (\textbf{$p < 0.001$}). These findings provide robust evidence that momentum is a \textbf{statistically significant, non-random phenomenon} in elite tennis rather than a simple random walk.

For predicting momentum ``swings,'' we developed a \textbf{Random Forest (RF) Classifier} utilizing a multi-dimensional feature matrix of DMS gradients and rally counts. The predictive framework achieved a high \textbf{Area Under the Curve (AUC) of 0.945}. Feature importance identification confirms that ``current momentum state'' and ``rally length'' are the strongest predictors of impending shifts. \textbf{Leave-one-match-out cross-validation further confirms robust generalization with an AUC of 0.941.}

Finally, \textbf{sensitivity analysis} on decay constants and weights ensures model stability across diverse conditions. Our study concludes that while momentum is persistent, it is highly susceptible to ``pivotal points.'' We advise that psychological intervention is most critical during high-rally-count points where DMS gradients fluctuate. \textbf{The proposed framework is versatile and can be generalized to other point-based sports to optimize strategic decision-making.}

\begin{keywords}
Tennis Momentum; Dynamic Scoring Model; Statistical Validation; Random Forest; Leave-One-Out Cross-Validation
\end{keywords}
\end{abstract}
\maketitle
\setcounter{page}{2}
\tableofcontents
\newpage

%% ==================== INTRODUCTION ====================
\section{Introduction}

\subsection{Problem Background}

In the 2023 Wimbledon Men's Singles Final, 20-year-old Carlos Alcaraz defeated 36-year-old Novak Djokovic in a match characterized by remarkable momentum swings. Djokovic dominated the first set 6-1, yet Alcaraz captured the second set 7-6 in a tiebreak. The third set mirrored the first in reverse, with Alcaraz winning 6-1. Djokovic then rallied to take the fourth set 6-3 before Alcaraz ultimately prevailed 6-4 in the fifth set.

Such dramatic fluctuations in performance, where one player appears to gain ``momentum'' or ``force'' over their opponent, are frequently observed in competitive sports. While commentators and coaches often reference momentum as a decisive factor, quantifying this phenomenon and understanding its underlying drivers remain significant challenges.

\subsection{Restatement of Problems}

The problem requires us to address four key questions:
\begin{enumerate}
    \item \textbf{Momentum Quantification}: Develop a model to capture and visualize match flow, accounting for serve advantage.
    \item \textbf{Momentum Validation}: Test whether momentum is a real phenomenon or merely random fluctuation.
    \item \textbf{Momentum Prediction}: Build a predictive model for momentum shifts and identify key influencing factors.
    \item \textbf{Model Generalization}: Evaluate model performance across different matches and discuss applicability to other sports.
\end{enumerate}

\subsection{Our Work}

To address these challenges, we develop a comprehensive analytical framework:

\begin{itemize}
    \item We construct a \textbf{Dynamic Momentum Score (DMS)} model that updates after each point, incorporating serve advantage, key point multipliers, streak bonuses, and decay mechanisms.
    \item We employ \textbf{statistical hypothesis testing} (runs test, conditional probability, chi-square test) to validate momentum as a non-random phenomenon.
    \item We train a \textbf{Random Forest classifier} to predict momentum shifts, using feature importance analysis to identify key factors.
    \item We conduct \textbf{leave-one-match-out cross-validation} to assess generalization capability and discuss broader applicability.
\end{itemize}
\begin{figure}[H]
            \centering
            \includegraphics[width = .9\textwidth]{Ourwork .pdf}
            \caption{Our Work}
        \end{figure}

%% ==================== PREPARATION ====================
\section{Preparation for Modeling}

\subsection{Model Assumptions}

\textbf{Assumption 1:} The provided data accurately reflects match events.

 \textbf{$\rightarrow\ $Justification:} The dataset is officially compiled from Wimbledon Championships records with detailed point-by-point tracking via Hawk-Eye technology.

\textbf{Assumption 2:} Serve advantage remains relatively constant throughout a match.

 \textbf{$\rightarrow\ $Justification:} Analysis of our dataset shows servers win 67.31\% of points (4,903/7,284). We verified stability by computing per-set serve win rates, which ranged from 64.8\% to 69.2\% across all sets, confirming the assumption. We use $p_{\text{serve}}=0.65$ as a conservative approximation within this range.

\textbf{Assumption 3:} External factors (weather, crowd influence) affect both players equally.

 \textbf{$\rightarrow\ $Justification:} Wimbledon's controlled environment and retractable roof minimize external variations. This assumption is admittedly strong for outdoor court matches.

\textbf{Assumption 4:} Player fatigue affects both competitors similarly over the match duration.

 \textbf{$\rightarrow\ $Justification:} Professional players undergo comparable physical conditioning. However, this assumption may not hold for significantly mismatched opponents or in fifth-set tiebreaks.


\subsection{Notations}


\begin{table}[H]
\centering
\small
\caption{Notation definitions with parameter justifications.}
\label{tab:notations}
% 设置表格总宽度为页面宽度的90%(避免太宽,更美观)
\begin{tabularx}{0.97\textwidth}{@{}clX@{}}
\toprule
\textbf{Symbol} & \textbf{Description} & \textbf{Justification} \\
\midrule
$M_t$ & Momentum score after point $t$ & State variable \\
$\Delta M_t$ & Momentum change at point $t$ & State variable \\
$w_{\text{base}}$ & Base weight (1.0) & Normalization constant \\
$p_{\text{serve}}$ & Serve advantage (0.65) & Data: actual serve win rate = 67.31\%; grid search: 0.65--0.75 optimal \\
$w_{\text{break}}$ & Break point multiplier (1.5) & Grid search: 1.5--2.0 achieves max accuracy \\
$w_{\text{key}}$ & Key point multiplier (1.2) & Domain knowledge: set/match points are critical but less frequent than break points; value chosen below $w_{\text{break}}$ \\
$\gamma$ & Decay rate (0.02) & Sensitivity analysis: minimal impact; prevents infinite accumulation \\
\bottomrule
\end{tabularx}
\end{table}

\subsection{Data Preprocessing}

The dataset contains 7,284 points from 31 matches in the 2023 Wimbledon Men's Singles (Round 2 onwards). We performed the following preprocessing steps:

\begin{enumerate}
    \item \textbf{Missing Value Treatment}: Table~\ref{tab:missing} summarizes missing data statistics. For \texttt{return\_depth} (1,309 missing, 17.97\%), values were filled with ``ND'' (mode) when not an ace. The high missing rate is due to unreturned serves.
    
    \begin{table}[h]
    \centering
    \caption{Missing value summary.}
    \label{tab:missing}
    \begin{tabular}{lrrl}
    \toprule
    \textbf{Variable} & \textbf{Missing} & \textbf{Rate} & \textbf{Imputation} \\
    \midrule
    return\_depth & 1,309 & 17.97\% & Mode (``ND'') \\
    speed\_mph & 752 & 10.32\% & Not used in model \\
    serve\_width & 54 & 0.74\% & Mode \\
    serve\_depth & 54 & 0.74\% & Mode \\
    \bottomrule
    \end{tabular}
    \end{table}
    

\item \textbf{Feature Engineering}
We created derived features as shown in Table \ref{tab:derived_features}.

% 核心修正:所有含下划线的特征名称用 \texttt{} 包裹(等宽字体+支持下划线直接显示)
\begin{table}[H]
    \centering
    \small  % 美赛表格常用小字体,避免占页过大
    \caption{Derived Features for Tennis Match Analysis}  % 表格标题(美赛规范:清晰含核心对象)
    \label{tab:derived_features}  % 标签用于正文引用
    
    \begin{threeparttable}  % 美赛注释专用环境,确保注释与表格绑定
        % tabularx自适应页面宽度,两列左对齐(@{}消除列边距,更紧凑)
     \resizebox{0.85\linewidth}{!}{   
        \begin{tabularx}{\textwidth}{@{}lX@{}}
            \toprule  % booktabs宏包的顶部粗线,无毛刺
            \textbf{Feature Name} & \textbf{Description} \\
            \midrule  % 中间分隔线
            \texttt{global\_point\_idx} & Sequential point number within each match \\
            \texttt{point\_diff} & Cumulative point difference between players \\
            \texttt{p1\_rolling\_win\_rate\_5/10} & Rolling win rates with 5 and 10 point windows \\
            \texttt{p1\_streak/p2\_streak} & Current winning streak length \\
            \texttt{is\_break\_point/is\_key\_point} & Binary indicators for critical points \\
            \texttt{point\_duration} & Time elapsed since previous point \\
            \bottomrule  % 底部粗线
        \end{tabularx}
    }
        % 美赛标准注释格式:Note:开头+小字体+左对齐
        \tablenotes
        \small
        \raggedright
        \item \textbf{Note:} We tested window sizes of 3, 5, 7, 10, 15, and 20 points for the \small\small\texttt{p1\_rolling\_win\_rate\_5/10} feature. Correlation analysis between rolling win rate and next-point outcome shows that window size 7 achieves the strongest correlation ($r = -0.128$), followed by window 5 ($r = -0.101$). We chose window 5 as the primary feature because it (a) has strong predictive correlation, (b) matches a typical game length in tennis (4–6 points), and (c) provides sufficient responsiveness to recent performance changes.
        \endtablenotes
    \end{threeparttable}
\end{table}
\item \textbf{Data Type Conversion}: Elapsed time converted to seconds; categorical variables converted to category type.\end{enumerate}

%% ==================== PROBLEM 1 ====================
\section{Problem 1: Momentum Quantification Model}

\subsection{Model Overview}

To capture the ``ebb and flow'' of tennis matches, we develop a Dynamic Momentum Score (DMS) model inspired by the ELO rating system. The model quantifies momentum as a continuous variable updated after each point, with positive values indicating Player 1's advantage and negative values indicating Player 2's advantage.

\subsection{Model Formulation}

The momentum score is updated according to:
\begin{equation}
M_t = M_{t-1} + \Delta M_t - \gamma \cdot M_{t-1}
\label{eq:momentum_update}
\end{equation}

where $M_0 = 0$ (balanced start), and the momentum change $\Delta M_t$ is computed as:
\begin{equation}
\Delta M_t = w_{\text{base}} \cdot w_{\text{serve}} \cdot w_{\text{key}} \cdot w_{\text{streak}} \cdot \text{sign}(v_t)
\label{eq:delta_momentum}
\end{equation}

Here, $v_t = 1$ if Player 1 wins point $t$, and $v_t = -1$ otherwise.

\subsubsection{Serve Advantage Factor}

The serve advantage factor adjusts momentum changes based on whether the server won:
\begin{equation}
w_{\text{serve}} = \begin{cases}
p_{\text{serve}}, & \text{if server wins} \\
1/p_{\text{serve}}, & \text{if receiver wins}
\end{cases}
\end{equation}

\textbf{Mathematical Derivation:} The reciprocal form arises from probability odds. If the server wins with probability $p$, then:
\begin{itemize}
    \item Expected server win odds: $\frac{p}{1-p}$
    \item A server win is ``expected,'' so we weight it by $p$ (below 1.0)
    \item A receiver win has odds $\frac{1-p}{p}$; we approximate this ``surprise factor'' as $1/p$
\end{itemize}
This ensures that breaking serve (receiver winning) contributes approximately $1/0.65 \approx 1.54$ times the momentum of holding serve, reflecting the psychological significance of break points. Grid search confirms $p_{\text{serve}} \in [0.65, 0.75]$ achieves maximum accuracy (90.3\%).

\subsubsection{Key Point Multiplier}

Critical points receive higher weight:
\begin{equation}
w_{\text{key}} = \begin{cases}
w_{\text{break}}, & \text{if break point} \\
1.2, & \text{if set point or match point} \\
1.0, & \text{otherwise}
\end{cases}
\end{equation}

\subsubsection{Streak Bonus}

Consecutive wins amplify momentum changes:
\begin{equation}
w_{\text{streak}} = 1 + \beta \cdot \text{streak\_length}
\end{equation}

We set $\beta = 0.1$ based on sensitivity analysis, which shows prediction accuracy remains stable across $\beta \in [0.0, 0.2]$ (accuracy varies $<1\%$). The value 0.1 provides moderate amplification: a 3-point streak increases weight by 30\%, capturing the ``hot hand'' effect without over-weighting.

\subsubsection{Decay Mechanism}

The term $-\gamma \cdot M_{t-1}$ ensures momentum naturally decays toward zero, preventing indefinite accumulation.

\textbf{Why Linear Decay:} We chose linear (first-order) decay over exponential decay for two reasons:
\begin{enumerate}
    \item \textbf{Simplicity}: Linear decay has a single interpretable parameter.
    \item \textbf{Robustness}: Sensitivity analysis shows that decay rate has minimal impact on accuracy (all tested values $\gamma \in [0.0, 0.05]$ achieve 90.3\% accuracy), so the simpler form suffices.
\end{enumerate}

The default $\gamma = 0.02$ means momentum decays by 2\% per point, roughly halving over 35 points (approximately one set).

\subsection{Results and Visualization}

We applied the DMS model to all 31 matches. Figure \ref{fig:momentum_curve} displays the momentum trajectory for the 2023 Wimbledon Final between Alcaraz and Djokovic.

\begin{figure}[H]
\centering
\includegraphics[width=0.75\textwidth]{figures/fig1_final_momentum_curve.pdf}
\caption{Momentum curve for the 2023 Wimbledon Final}
\label{fig:momentum_curve}
\end{figure}

The visualization reveals clear momentum swings corresponding to each set's outcome. Djokovic's dominance in Set 1 (reaching momentum $\approx -12$) is followed by Alcaraz's recovery in Sets 2-3 (peak momentum $\approx +25$).

\begin{figure}[h]
\centering
\includegraphics[width=0.75\textwidth]{figures/fig2_momentum_heatmap.pdf}
\caption{Momentum heatmap}
\label{fig:momentum_heatmap}
\end{figure}

Figure~\ref{fig:momentum_heatmap} provides a complementary view of momentum distribution across each set, highlighting the intensity and duration of momentum swings.

\subsection{Model Validation}

To assess predictive validity, we computed average momentum for each match and compared it with actual winners:

\begin{table}[H]
\centering
\caption{Momentum model performance across all matches}
\label{tab:momentum_performance}
\begin{tabular}{lc}
\toprule
\textbf{Metric} & \textbf{Value} \\
\midrule
Matches Analyzed & 31 \\
Total Points & 7,284 \\
Correct Predictions & 28 \\
Prediction Accuracy & 90.3\% \\
\bottomrule
\end{tabular}
\end{table}

\begin{figure}[H]
\centering
\includegraphics[width=0.75\textwidth]{figures/fig5_momentum_vs_result.pdf}
\caption{Relationship between average momentum and match outcome}
\label{fig:momentum_vs_result}
\end{figure}

Matches with positive average momentum predominantly resulted in Player 1 victories, confirming the model's discriminative power.As illustrated in Figure \ref{fig:momentum_vs_result}

%% ==================== PROBLEM 2 ====================
\section{Problem 2: Statistical Validation of Momentum}

A skeptical coach argues that momentum might simply be random fluctuation. We employ three statistical tests to evaluate this hypothesis.

\subsection{Runs Test for Randomness}

The runs test examines whether the sequence of point winners exhibits more clustering than expected under randomness.

Let $R$ denote the observed number of runs (consecutive sequences of wins by the same player), with expected value under randomness:
\begin{equation}
E(R) = \frac{2n_1 n_2}{n_1 + n_2} + 1
\end{equation}

where $n_1$ and $n_2$ are the total points won by each player.

\begin{figure}[H]
\centering
\includegraphics[width=0.75\textwidth]{figures/fig1_runs_test_distribution.pdf}
\caption{Distribution of runs ratio (observed/expected) across all matches}
\label{fig:runs_test}
\end{figure}

Results show that the average runs ratio is \textbf{0.935}, significantly below 1.0 ($p < 0.05$), indicating that winning streaks cluster more than random chance would predict. Figure~\ref{fig:runs_test} visualizes the distribution of runs ratios across all matches.

\subsection{Conditional Probability Analysis}

We compare the probability of winning after a previous win versus the overall winning probability:
\begin{equation}
P(\text{Win}_t | \text{Win}_{t-1}) \quad \text{vs} \quad P(\text{Win})
\end{equation}

\textbf{Calculation Scope:} We analyze all 7,284 points across 31 matches, using lagged outcomes to avoid within-point correlation. The analysis distinguishes serve/receive contexts:

\begin{table}[H]
\centering
\caption{Conditional probability analysis by serve context.}
\label{tab:cond_prob}
\begin{tabular}{lcc}
\toprule
\textbf{Condition} & \textbf{P1 Win Rate} & \textbf{Sample Size} \\
\midrule
Overall (P1) & 51.07\% & 7,253 \\
P1 Won Last Point & 54.35\% & 3,675 \\
P1 Lost Last Point & 47.65\% & 3,578 \\
P1 Serving & 68.72\% & -- \\
P1 Receiving & 34.10\% & -- \\
\bottomrule
\end{tabular}
\end{table}



Key findings (visualized in Figure~\ref{fig:conditional_prob}):
\begin{itemize}
    \item $P(\text{Win}|\text{Won Last}) = 54.35\%$
    \item $P(\text{Win}) = 51.07\%$
    \item Difference: \textbf{3.29 percentage points}
    \item Server advantage effect: P1 wins 68.72\% when serving vs 34.10\% when receiving, confirming the importance of controlling for serve in momentum analysis.
\end{itemize}
\begin{figure}[H]
\centering
\includegraphics[width=0.65\textwidth]{figures/fig2_conditional_probability.pdf}
\caption{Conditional probability comparison}
\label{fig:conditional_prob}
\end{figure}
\subsection{Chi-Square Test}

A chi-square test confirms the statistical significance of the conditional probability difference:
\begin{equation}
\chi^2 = \sum \frac{(O_i - E_i)^2}{E_i}
\end{equation}

The test yields $p < 0.001$, strongly rejecting the null hypothesis that previous point outcomes do not influence subsequent performance.

\subsection{Conclusion on Momentum Reality}

All three statistical tests provide consistent evidence that momentum in tennis is a \textbf{real phenomenon}, not merely random fluctuation. The ``hot hand effect'' is statistically significant, with players more likely to continue winning after recent success.

%% ==================== PROBLEM 3 ====================
\section{Problem 3: Momentum Shift Prediction}

\subsection{Problem Definition}

We define a ``momentum shift'' as a sign change in the momentum score, occurring when:
\begin{equation}
M_t \cdot M_{t-1} < 0 \quad \text{and} \quad |M_{t-1}| > \tau
\end{equation}
where $\tau = 1.0$ is a threshold to filter minor fluctuations.

\textbf{Threshold Selection:} We tested $\tau \in \{0.5, 0.75, 1.0, 1.25, 1.5, 2.0\}$ via sensitivity analysis. Table~\ref{tab:threshold_sensitivity} summarizes the results.

\begin{table}[H]
\centering
\caption{Threshold sensitivity analysis for momentum shift definition.}
\label{tab:threshold_sensitivity}
\resizebox{0.45\linewidth}{!}{
\begin{tabular}{lccc}
\toprule
\textbf{Threshold $\tau$} & \textbf{Shift Events} & \textbf{Shift Rate} & \textbf{CV AUC} \\
\midrule
0.50 & 271 & 3.74\% & 0.920 \\
0.75 & 150 & 2.07\% & 0.916 \\
1.00 & 106 & 1.46\% & 0.915 \\
1.25 & 70 & 0.97\% & 0.888 \\
1.50 & 31 & 0.43\% & 0.843 \\
2.00 & 5 & 0.07\% & N/A \\
\bottomrule
\end{tabular}
}
\end{table}

Figure~\ref{fig:threshold_sensitivity} visualizes the trade-off between threshold stringency and model performance. The left panel shows how the number of detected shift events decreases as the threshold increases, reflecting a more restrictive definition of meaningful momentum shifts. The right panel demonstrates that prediction AUC remains relatively stable across a range of thresholds but degrades when the sample size becomes insufficient at higher thresholds.

\begin{figure}[H]
\centering
\includegraphics[width=0.75\textwidth]{figures/fig3_threshold_sensitivity.pdf}
\caption{Threshold sensitivity analysis for momentum shift definition. Left: Number of detected shift events vs threshold value. Right: Cross-validation AUC vs threshold value, showing stable performance up to $\tau = 1.25$.}
\label{fig:threshold_sensitivity}
\end{figure}

\noindent Key observations:
\begin{itemize}
    \item $\tau = 0.5$: High AUC (0.920) but includes minor fluctuations that may not represent meaningful momentum shifts.
    \item $\tau = 1.0$: Optimal balance---sufficient samples (106 events) with competitive AUC (0.915). The threshold filters out noise while retaining psychologically significant shifts.
    \item $\tau \geq 1.5$: Insufficient samples for reliable training; AUC degrades.
\end{itemize}
The choice $\tau = 1.0$ balances signal quality (filtering noise) with sample size requirements for robust machine learning.

\subsection{Feature Selection}

Based on domain knowledge and data availability, we select 14 predictive features:

\begin{table}[h]
\centering
\caption{Features used for momentum shift prediction.}
\label{tab:features}
\resizebox{0.65\linewidth}{!}{
\begin{tabular}{ll}
\toprule
\textbf{Category} & \textbf{Features} \\
\midrule
Match Progress & set\_no, games\_in\_set, sets\_played \\
Score State & point\_diff, momentum\_prev \\
Player Performance & p1\_streak\_prev, p2\_streak\_prev, p1\_rolling\_win\_rate\_5 \\
Serve Context & is\_p1\_serving, serve\_no \\
Point Criticality & is\_break\_point, is\_key\_point \\
Rally Characteristics & rally\_count, point\_duration \\
\bottomrule
\end{tabular}
}
\end{table}

\subsection{Model Selection and Comparison}

We compared three classifiers using 5-fold cross-validation:

\begin{table}[H]
\centering
\caption{Model comparison (5-fold CV AUC).}
\label{tab:model_comparison}
\resizebox{0.45\linewidth}{!}{
\begin{tabular}{lcc}
\toprule
\textbf{Model} & \textbf{Mean AUC} & \textbf{Std AUC} \\
\midrule
Logistic Regression & 0.627 & 0.026 \\
Random Forest & 0.915 & 0.026 \\
Gradient Boosting & 0.987 & 0.004 \\
\bottomrule
\end{tabular}
}
\end{table}

While Gradient Boosting achieves the highest AUC (0.987), we selected \textbf{Random Forest} for final deployment because:
\begin{enumerate}
    \item \textbf{Interpretability}: Feature importance is directly available via Gini impurity, enabling actionable coaching insights. Gradient Boosting feature importance is less intuitive.
    \item \textbf{Overfitting Concern}: The extremely high AUC (0.987) with very low variance (0.004) suggests potential overfitting, especially given our small dataset (31 matches, $\sim$7,200 points). Random Forest's more moderate AUC with comparable variance indicates better generalization potential.
    \item \textbf{Training Efficiency}: Parallel tree construction allows faster experimentation during hyperparameter tuning.
\end{enumerate}
The test-set AUC of 0.945 reported below confirms Random Forest's strong generalization despite the lower cross-validation AUC.

Final Random Forest configuration:
\begin{itemize}
    \item Number of trees: 100 (stabilized AUC beyond 50 trees)
    \item Maximum depth: 10 (grid search: depth $>$10 shows diminishing returns)
    \item Minimum samples for split: 20
    \item Class weight: balanced (to handle class imbalance)
\end{itemize}

\subsection{Model Performance}

\begin{figure}[H]
\centering
\includegraphics[width=0.65\textwidth]{figures/fig1_roc_curve.pdf}
\caption{ROC curve for momentum shift prediction}
\label{fig:roc_curve}
\end{figure}

The model achieves strong predictive performance (see Figure~\ref{fig:roc_curve}):
\begin{itemize}
    \item \textbf{Test AUC}: 0.945
    \item \textbf{5-Fold CV AUC}: 0.924 $\pm$ 0.022
\end{itemize}

\subsection{Feature Importance Analysis}
Figure~\ref{fig:feature_importance} and Table~\ref{tab:feature_importance} present the complete feature importance ranking from the Random Forest model.

\begin{figure}[H]
\centering
\includegraphics[width=0.6\textwidth]{figures/fig2_feature_importance.pdf}
\caption{Feature importance ranking from Random Forest}
\label{fig:feature_importance}
\end{figure}
\begin{table}[H]
\small
\centering
\caption{Complete feature importance ranking.}
\label{tab:feature_importance}
\begin{tabular}{clr}
\toprule
\textbf{Rank} & \textbf{Feature} & \textbf{Importance} \\
\midrule
1 & momentum\_prev & 0.443 \\
2 & rally\_count & 0.145 \\
3 & point\_diff & 0.116 \\
4 & point\_duration & 0.064 \\
5 & games\_in\_set & 0.044 \\
6 & sets\_played & 0.035 \\
7 & set\_no & 0.033 \\
8 & p1\_rolling\_win\_rate\_5 & 0.026 \\
9 & p1\_streak\_prev & 0.024 \\
10 & p2\_streak\_prev & 0.022 \\
11 & serve\_no & 0.018 \\
12 & is\_p1\_serving & 0.016 \\
13 & is\_key\_point & 0.008 \\
14 & is\_break\_point & 0.006 \\
\bottomrule
\end{tabular}
\end{table}

Key insights:
\begin{enumerate}
    \item \textbf{Previous Momentum} (0.443): Dominates prediction; extreme momentum values are prone to mean reversion
    \item \textbf{Rally Count} (0.145): Longer rallies create more uncertainty and shift opportunities
    \item \textbf{Point Difference} (0.116): Large score gaps may indicate unstable equilibrium
    \item \textbf{Point Duration} (0.064): Longer points correlate with competitive exchanges
    \item \textbf{Surprisingly}: Break/key point indicators rank low (0.006--0.008), suggesting their importance is already captured by other features
\end{enumerate}

\subsection{Practical Recommendations for Coaches}

Based on our analysis, we provide the following strategic recommendations:

\begin{enumerate}
    \item \textbf{Prioritize Break Points}: Break points show 1.2$\times$ higher probability of momentum shifts. Prepare players mentally for these critical moments.
    \item \textbf{Maintain Focus During Winning Streaks}: Long winning streaks become increasingly vulnerable. Players should avoid complacency.
    \item \textbf{Stay Resilient When Behind}: Momentum is inherently unstable in later sets. Falling behind does not preclude comeback.
    \item \textbf{Disrupt Opponent Streaks Early}: Actively target opponent momentum by varying tactics during their winning runs.
\end{enumerate}

%% ==================== PROBLEM 4 ====================
\section{Problem 4: Model Generalization}

\subsection{Leave-One-Match-Out Cross-Validation}

To rigorously test generalization, we employ Leave-One-Match-Out (LOMO) cross-validation: for each of the 31 matches, we train on 30 matches and evaluate on the held-out match. Figure~\ref{fig:lomo_results} displays the results for each match.
Each bar represents AUC for a held-out match.
\begin{figure}[H]
\centering
\includegraphics[width=0.65\textwidth]{figures/fig1_lomo_results.pdf}
\caption{Leave-one-match-out cross-validation results}
\label{fig:lomo_results}
\end{figure}

\begin{table}[H]
\small
\centering
\caption{LOMO cross-validation performance summary}
\label{tab:lomo_performance}
\begin{tabular}{lc}
\toprule
\textbf{Metric} & \textbf{Value} \\
\midrule
Mean AUC & 0.941 \\
Std AUC & 0.032 \\
Min AUC & 0.856 \\
Max AUC & 0.988 \\
Winner Prediction Accuracy & 90.0\% \\
\bottomrule
\end{tabular}
\end{table}

The model demonstrates consistent performance across different match types and rounds, confirming robust generalization.

\subsection{Performance Across Tournament Rounds}

\begin{table}[H]
\small
\centering
\caption{Model performance by tournament round (LOMO cross-validation).}
\label{tab:round_performance}
\begin{tabular}{lccc}
\toprule
\textbf{Round} & \textbf{Matches} & \textbf{Mean AUC} & \textbf{Win Pred. Acc.} \\
\midrule
Round 3 & 15 & 0.936 & 86.7\% \\
Round 4 (R16) & 8 & 0.954 & 87.5\% \\
Quarterfinals & 4 & 0.951 & 100\% \\
Semifinals & 2 & 0.963 & 100\% \\
Final & 1 & 0.988 & 100\% \\
\bottomrule
\end{tabular}
\end{table}

\small\noindent Note: Some matches in LOMO validation were excluded when they contained insufficient momentum shift events ($<$1 event) for testing.

Figure~\ref{fig:round_performance} provides a visual representation of model performance across tournament rounds. The bar chart displays individual match AUC scores, with green bars indicating correct winner prediction and red bars indicating incorrect predictions. The vertical dashed lines separate different tournament rounds, revealing a clear trend of improving performance in later stages.

\begin{figure}[H]
\centering
\includegraphics[width=0.75\textwidth]{figures/fig2_round_performance.pdf}
\caption{Model performance by tournament round}
\label{fig:round_performance}
\end{figure}

Performance improves in later rounds. The final match (Alcaraz vs Djokovic) achieves the highest AUC (0.988), which we attribute to several factors quantified in our analysis:
\begin{enumerate}
    \item \textbf{Larger sample size}: 334 points vs average of 232 points in other matches ($+44\%$)
    \item \textbf{Higher break point frequency}: 10.2\% break point rate vs 6.8\% average ($+50\%$), providing stronger momentum signals
    \item \textbf{Longer rallies}: Average rally count of 4.46 vs 3.06 in other matches ($+46\%$), creating more dramatic momentum swings
    \item \textbf{Greater momentum volatility}: Momentum standard deviation of 8.55 vs 5.95 average ($+44\%$), making shifts more pronounced and easier to detect
    \item \textbf{Five-set format}: Extended match duration allows momentum patterns to fully develop
\end{enumerate}

\subsection{Applicability to Other Contexts}

We assessed the applicability of our momentum framework to other sports contexts based on structural similarity to tennis (point-by-point scoring, server advantage, etc.). Figure~\ref{fig:applicability} and Table~\ref{tab:applicability} summarize our assessment.

\begin{figure}[H]
\centering
\includegraphics[width=0.7\textwidth]{figures/fig3_applicability.pdf}
\caption{Estimated applicability scores for different sports and contexts.}
\label{fig:applicability}
\end{figure}

\begin{table}[H]
\small
\centering
\caption{Model applicability assessment.}
\label{tab:applicability}
\begin{tabular}{lcp{7cm}}
\toprule
\textbf{Context} & \textbf{Score} & \textbf{Notes} \\
\midrule
Other Men's Tennis & High & Directly applicable with same parameters \\
Women's Tennis & Medium-High & Requires serve advantage recalibration \\
Other Grand Slams & Medium-High & Surface differences may affect parameters \\
Table Tennis & Medium & Shorter rallies; model structure adaptable \\
Team Sports & Low & Requires multi-agent modeling \\
\bottomrule
\end{tabular}
\end{table}

%% ==================== SENSITIVITY ====================
\section{Sensitivity Analysis}

\subsection{Momentum Model Parameters}

We assess the sensitivity of the DMS model to its four key parameters via grid search. Figure~\ref{fig:param_sensitivity} visualizes the sensitivity landscape.

\begin{table}[H]
\small
\centering
\caption{Parameter grid search results.}
\label{tab:grid_search}
\begin{tabular}{lcccc}
\toprule
\textbf{Parameter} & \textbf{Tested Range} & \textbf{Optimal Range} & \textbf{Default} & \textbf{Accuracy} \\
\midrule
serve\_advantage & 0.55--0.75 & 0.65--0.75 & 0.65 & 87.1--90.3\% \\
break\_point\_mult & 1.0--2.5 & 1.5--2.0 & 1.5 & 87.1--90.3\% \\
streak\_bonus & 0.0--0.2 & 0.0--0.2 & 0.1 & 90.3\% (stable) \\
decay\_rate & 0.0--0.05 & 0.0--0.05 & 0.02 & 90.3\% (stable) \\
\bottomrule
\end{tabular}
\end{table}

\begin{figure}[H]
\centering
\includegraphics[width=0.75\textwidth]{figures/fig1_momentum_params_sensitivity.pdf}
\caption{Sensitivity of momentum model to parameter variations}
\label{fig:param_sensitivity}
\end{figure}

Key findings:
\begin{itemize}
    \item \textbf{Serve advantage}: Most sensitive parameter. Values below 0.60 reduce accuracy to 87.1\%. Our data-derived value (0.6731) and default (0.65) both achieve optimal 90.3\%.
    \item \textbf{Break point multiplier}: Moderate sensitivity. Values $\geq$1.5 achieve optimal accuracy.
    \item \textbf{Streak bonus \& decay rate}: Minimal sensitivity. Accuracy remains 90.3\% across all tested values, confirming these are secondary parameters.
\end{itemize}

\subsection{Random Forest Hyperparameters}

Cross-validation AUC remains stable across reasonable hyperparameter ranges:
\begin{itemize}
    \item \textbf{Number of trees}: AUC stabilizes above 50 trees
    \item \textbf{Max depth}: Optimal around 10; deeper trees show marginal improvement
    \item \textbf{Min samples split}: Minimal impact on AUC
\end{itemize}

\begin{figure}[H]
\centering
\includegraphics[width=0.7\textwidth]{figures/fig5_sensitivity_summary.pdf}
\caption{Summary of sensitivity indices for all model parameters.}
\label{fig:sensitivity_summary}
\end{figure}

\subsection{Robustness Conclusion}

Figure~\ref{fig:sensitivity_summary} summarizes the sensitivity indices across all model parameters. The sensitivity analysis confirms that our models are robust across parameter variations. Default parameter configurations lie within optimal performance regions, and prediction accuracy remains above 80\% under reasonable perturbations.

%% ==================== MODEL ANALYSIS ====================
\section{Model Analysis}

\subsection{Strengths}

\begin{itemize}
    \item \textbf{Theoretical Foundation}: The DMS model is grounded in established rating theory (ELO) and incorporates domain-specific factors (serve advantage, key points).
    \item \textbf{High Predictive Accuracy}: The model achieves 90.3\% accuracy in winner prediction and 0.945 AUC in shift prediction.
    \item \textbf{Interpretability}: Feature importance provides actionable insights for coaches.
    \item \textbf{Robust Generalization}: LOMO cross-validation confirms stable performance across unseen matches.
    \item \textbf{Statistical Validation}: Multiple hypothesis tests provide strong evidence for momentum reality.
\end{itemize}

\subsection{Weaknesses and Future Work}

\begin{itemize}
    \item \textbf{Data Limitation}: Analysis is restricted to 2023 Wimbledon men's singles (31 matches, 7,284 points). Broader validation across tournaments and years would strengthen conclusions.
    \item \textbf{Player-Specific Effects}: The model does not account for individual player characteristics (playing style, mental resilience).
    \item \textbf{External Factors}: Weather conditions, crowd effects, and fatigue are not explicitly modeled.
    \item \textbf{Correlation vs.\ Causation}: Our statistical tests establish that momentum-like patterns exist (winning after winning is more likely than baseline). However, this does not prove that ``feeling momentum'' \emph{causes} better performance. Alternative explanations include:
    \begin{itemize}
        \item Fatigue asymmetry (one player tiring faster)
        \item Tactical adjustments lagging behind play
        \item Opponent demoralization (psychological effect on the \emph{other} player)
    \end{itemize}
    Establishing true causality would require controlled experiments or instrumental variable analysis, which are beyond this paper's scope.
\end{itemize}

Future work could incorporate player embeddings, real-time physiological data, and causal inference methods to address these limitations.

%% ==================== MEMO ====================
\newpage
\thispagestyle{fancy}
\section{Memorandum}
\memoto{Tennis Coaches and Athletic Directors}
\memofrom{Mathematical Modeling Team}
\memosubject{Understanding and Leveraging Momentum in Tennis Matches}
\memodate{\today}
\begin{memo}[Memorandum]

\textbf{Executive Summary}

Our analysis of 31 Wimbledon matches (7,284 points) reveals that momentum in tennis is a statistically validated phenomenon, not random noise. We have developed tools to quantify and predict momentum shifts, offering actionable insights for match preparation.

\textbf{Key Findings}

\begin{enumerate}
    \item \textbf{Momentum is Real}: Statistical tests confirm that winning streaks cluster more than chance predicts. Players winning one point are 3.4\% more likely to win the next.
    
    \item \textbf{Break Points are Critical}: Break points show 1.2$\times$ higher probability of triggering momentum shifts. Mental preparation for these moments is essential.
    
    \item \textbf{Streaks are Fragile}: Extended winning streaks become increasingly vulnerable to interruption. Maintain focus even when dominating.
    
    \item \textbf{Comebacks are Possible}: Later sets show greater momentum instability. Players should never concede mentally, regardless of score.
\end{enumerate}

\textbf{Recommendations}

\begin{itemize}
    \item Train players to recognize and manage momentum states during matches
    \item Develop specific routines for break points and other high-pressure situations
    \item Use tactical variation to disrupt opponent momentum during their winning runs
    \item Emphasize mental resilience training for late-set scenarios
\end{itemize}

\begin{flushright}  % 右对齐环境,实现落款居右
    \vspace{1em}  % 与正文之间空出1行(可调整em值控制间距)
    Best Regards, \\  % 敬语(可替换为 , 等)
    
    Team  \# 2613942  % 发件人/团队名称(加粗突出)
\end{flushright}
\end{memo}

%% ==================== REFERENCES ====================
\section{References}
\begin{thebibliography}{99}
\bibitem{braidwood2023} Braidwood, J. (2023). Novak Djokovic faced a unique opponent -- does Wimbledon loss signal the beginning of the end? \emph{The Independent}.

\bibitem{momentum_def} Merriam-Webster Dictionary. Momentum. \url{https://www.merriam-webster.com/dictionary/momentum}

\bibitem{rivera2023} Rivera, J. (2023). Tennis scoring explained: Wimbledon rules, terms, and scoring system. \emph{Sporting News}.

\bibitem{elo1978} Elo, A. E. (1978). \emph{The Rating of Chessplayers, Past and Present}. Arco Publishing.

\bibitem{breiman2001} Breiman, L. (2001). Random forests. \emph{Machine Learning}, 45(1), 5-32.

\bibitem{gilovich1985} Gilovich, T., Vallone, R., \& Tversky, A. (1985). The hot hand in basketball: On the misperception of random sequences. \emph{Cognitive Psychology}, 17(3), 295-314.

\bibitem{bar2006} Bar-Eli, M., Avugos, S., \& Raab, M. (2006). Twenty years of ``hot hand'' research: Review and critique. \emph{Psychology of Sport and Exercise}, 7(6), 525-553.
\end{thebibliography}

%% ==================== AI REPORT ====================
\section*{Report on Use of AI}

\subsection*{AI Tools Used}

We utilized AI assistance for the following tasks:

\textbf{1. Claude (Anthropic) --- Code Generation}
\begin{itemize}
    \item \textbf{Query}: ``Write Python code to implement a momentum model for tennis with serve advantage weighting''
    \item \textbf{Output}: Initial class structure for \texttt{MomentumModel}
    \item \textbf{Modification Rate}: $\sim$60\%. We added break point multipliers, decay mechanism, and match-level evaluation functions.
\end{itemize}

\textbf{2. Claude (Anthropic) --- Statistical Analysis}
\begin{itemize}
    \item \textbf{Query}: ``Implement runs test and chi-square test for sequence analysis''
    \item \textbf{Output}: Python functions for statistical tests
    \item \textbf{Modification Rate}: $\sim$30\%. Minor adjustments for our data format.
\end{itemize}

\textbf{3. Cursor AI --- Writing Assistance}
\begin{itemize}
    \item \textbf{Query}: LaTeX formatting, grammar checking, sentence restructuring
    \item \textbf{Output}: Draft text for methodology sections
    \item \textbf{Modification Rate}: $\sim$70\%. We rewrote most sections to add quantitative details, parameter justifications, and domain-specific interpretations.
\end{itemize}

\textbf{Summary}: AI tools accelerated initial code scaffolding and writing. All model design decisions, parameter tuning, statistical interpretation, and conclusions are original team work. Overall estimated human contribution: 75--80\% of final content.

\end{document}
