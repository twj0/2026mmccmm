\newpage
%附录
\appendix
%附录需重新起页,论文附录至少应包括参赛论文的所有源程序代码,如实际使用的软件名称、命令和编写的全部可运行的源程序(含EXCEL、SPSS等软件的交互命令);通常还应包括自主查阅使用的数据等资料。赛题中提供的数据不要放在附录。如果缺少必要的源程序或程序不能运行(或者运行结果与正文不符),可能会被取消评奖资格。如果确实没有源程序,也应在论文附录中明确说明“本论文没有源程序”。 	

% 设置附录部分的编号格式和目录深度
\renewcommand{\thesubsection}{\Alph{subsection}}
\setcounter{tocdepth}{0}  % 附录内只显示section级别在目录中

\section*{Appendix}
\addcontentsline{toc}{section}{Appendix}

% 考虑到美赛页数限制,不做额外的数据图表展示,若有需要可取消注释。
% \subsection{Detailed Figures and Tables} % 索引方式	 \ref{xxtb1}  \ref{xxtb2}  \ref{xxtb3}
% 				  % 详细数据图请参见附录  \ref{xxtb1}。
% \subsubsection{******Detailed Data Table}%采用不编号格式
% \label{xxtb1}

% \subsubsection{******Detailed Data Table}
% \label{xxtb2}

% \subsubsection{******Detailed Data Table}
% \label{xxtb3}


\subsection{Code Programs}
% 插入Python代码
\lstinputlisting[style=Python, caption={Python Example}]{code/code_example.py}

% 插入MATLAB代码
\lstinputlisting[style=MATLAB, caption={MATLAB Example}]{code/code_example.m}








% 美赛不需要提供支撑材料
% \section{Supporting Materials}

% \begin{enumerate}
% 	\item 	Supporting Material 1
	
% 	\item 	Supporting Material 2
	
% 	\item 	Supporting Material 3
% \end{enumerate}

% 恢复目录深度设置
\setcounter{tocdepth}{3}