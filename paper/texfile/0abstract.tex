\begin{abstract}%摘要+摘要+摘要+摘要+摘要


	%第一段——问题重述+简要思想:首先简要叙述所给问题的背景和动机,并分别分析每个小问题的特点(以下以三个问题为例)。根据这些特点说出自己的思想:针对于问题1,采用。。。。。。。。的方法解决;针对问题2用。。。。。。。。的方法解决;针对问题3用。。。。。。。。的方法解决。
	\lipsum[2]	
	
	
	%\textbf{}:加粗	
	\textbf{In Question 1}, \lipsum[1]	
	%第二段——模型建立及求解结果:介绍思想和模型: 对于问题1我们首先建立了。。。。。。。。模型I。首先利用。。。。。。,其次计算了。。。。。。,并借助。。。。。。数学算法和。。。。。。软件得出了。。。。。。结论。
	
	
	
	
	\textbf{In Question 2},
	%(第3段)	对于问题2我们用。。。。。。。。
	
	
	
	\textbf{In Question 3},
	%(第4段)	对于问题3我们用。。。。。。。。(模型的建立与求解结果的陈述中,思想、模型、软件和结果必须描述清晰,亮点详细说明需突出。针对不同问题可独立成段也可采用一段式仅用分号“;”分割,摘要只接受文字描述形式,不接受图表等其他方式)
	
	
	
	
	
	%(第5段)	优化结果及总结:在。。。。。。条件下,针对。。。。。。模型进行适当修改与优化,这种条件的改变可能来自你的一种猜想或建议。要注意合理性。此推广模型可以不深入研究,也可以没有具体结果。
	
	%注:字数300~600之间,需控制在一页;摘要中必须将具体方法、模型和所得结果写出来;摘要要求“总分总”,段开头可用“针对问题1,针对问题2,针对问题3..”或者“首先,然后,其次,最后”等词语进行有逻辑的论述。摘要是重中之重,必须严格执行!
	
	
	% 美赛论文中无需注明关键字。若您一定要使用,
	% 请将以下两行的注释号 '%' 去除,以使其生效
	\vspace{5pt}  %mm	毫米	1 mm = 2.845 pt   pt 点	1 pt = 0.351 mm
	\noindent\textbf{Keywords}: keyword 1,  keyword 2,  keyword 3,  keyword 4,  keyword 5
	%\keywords:关键词;\quad:空格
	%使用到的模型名称、方法名称、特别是亮点一定要在关键字里出现,3~5个较合适,用分号隔开
\end{abstract}
\maketitle  % 生成 Summary Sheet
