\documentclass[12pt]{article}  % 官方要求字号不小于 12 号,此处选择 12 号字体
% \linespread{1.1}
% \bibliographystyle{plain}
% 本模板不需要填写年份,以当前电脑时间自动生成
% 请在以下的方括号中填写队伍控制号
\usepackage[123456]{easymcm}  % 123456是队伍号 载入 EasyMCM 模板文件
\problem{C}  % 请在此处填写题号
\title{This is a Title }  % 标题


% \usepackage{mathptmx}  % 这是 Times 字体,中规中矩 
\usepackage{palatino}  % mathpazo 这palatino是 COMAP 官方杂志采用的更好看的 Palatino 字体,可替代以上的 mathptmx 宏包
\usepackage{pdfpages}
\usepackage{longtable}
\usepackage{tabu}
\usepackage{lastpage}
\usepackage{threeparttable}
\usepackage{xcolor}

\usepackage{url} % 支持网址链接,便于引用网络资源
\usepackage{cite} % 优化参考文献引用格式
\usepackage{graphicx} % 插入和管理图片
\usepackage{pdflscape} % 使页面横向显示,适合宽表格
\usepackage{lscape} % 旋转页面内容,适合横向表格或图片
\usepackage{tabularx} % 支持自动调整宽度的表格
\usepackage{afterpage} % 控制内容在下一页显示
\usepackage{longtable} % 支持跨页长表格

\usepackage[numbers,sort&compress]{natbib} % 参考文献排序与压缩显示

\usepackage[figuresright]{rotating} % 支持图表旋转


\usepackage{amsmath} % 数学公式增强
\usepackage{amssymb} % 更多数学符号
\usepackage{geometry} % 页面边距自定义
\usepackage{enumitem} % 列表样式自定义
\usepackage{caption} % 图表标题样式自定义
\usepackage{fancyhdr} % 页眉页脚自定义
\usepackage{titlesec} % 章节标题样式自定义


\usepackage{lipsum} % 生成随机文本

\usepackage{cleveref} % 智能引用宏包,支持 \cref 命令自动添加图表标签前缀

% 全局设置表格行高
\renewcommand{\arraystretch}{1.3}

\usepackage{listings} % 插入和高亮显示代码
% 设置代码样式
\lstset{
	numbers             =   left,               % 行号显示在左侧
	showspaces          =   false,              % 不显示空格符号
	numberstyle         =   \ttfamily,% 行号风格,小五号,等宽字体
	showstringspaces    =   false,              % 不显示字符串中的空格
	captionpos          =   t,                  % 标题位置在顶部
	frame               =   lrtb,               % 显示四周边框
	breaklines          =   true,               % 自动换行
	columns             =   fixed,              % 固定列宽
	basewidth           =   0.5em,              % 基础宽度
	caption             =   \raggedright,       % 标题左对齐
}

% Python 代码样式
\lstdefinestyle{Python}{
	language        =   Python, % 语言设为Python
	basicstyle      =   \ttfamily\color{black}, % 基本风格,小五号,等宽字体,黑色
	keywordstyle    =   \color{blue}\bfseries\itshape, % 关键字风格,蓝色加粗斜体
	stringstyle     =   \color{magenta}, % 字符串风格,洋红色
	commentstyle    =   \color{red}\itshape, % 注释风格,红色斜体
}


\usepackage{paralist}
\graphicspath{{img/}}          % 此处{img/}为相对路径,注意加上“/”
\let\itemize\compactitem
\let\enditemize\endcompactitem

\usepackage{multicol} % 多栏排版

\usepackage{booktabs} % 优美的表格线
\usepackage{hyperref} % 支持PDF跳转、超链接
\usepackage{subcaption} % 子图标题,多个图片并排显示
\usepackage {float}  % 更灵活地控制浮动体(如表格、图片)位置
\usepackage{array} % 增强表格功能
\hypersetup{hidelinks,
	colorlinks=true,
	allcolors=black,
	pdfstartview=Fit,
	breaklinks=true
}
\usepackage{siunitx} % 科学计数法和单位格式化




\newcommand{\upcite}[1]{\textsuperscript{\textsuperscript{\cite{#1}}}}


% 如需要修改题头(默认为 MCM/ICM),请使用以下命令(此处修改为 MCM)
\renewcommand{\contest}{MCM}

%文档开始
\begin{document}
	
	

	\begin{abstract}%摘要+摘要+摘要+摘要+摘要


	%第一段——问题重述+简要思想:首先简要叙述所给问题的背景和动机,并分别分析每个小问题的特点(以下以三个问题为例)。根据这些特点说出自己的思想:针对于问题1,采用。。。。。。。。的方法解决;针对问题2用。。。。。。。。的方法解决;针对问题3用。。。。。。。。的方法解决。
	\lipsum[2]	
	
	
	%\textbf{}:加粗	
	\textbf{In Question 1}, \lipsum[1]	
	%第二段——模型建立及求解结果:介绍思想和模型: 对于问题1我们首先建立了。。。。。。。。模型I。首先利用。。。。。。,其次计算了。。。。。。,并借助。。。。。。数学算法和。。。。。。软件得出了。。。。。。结论。
	
	
	
	
	\textbf{In Question 2},
	%(第3段)	对于问题2我们用。。。。。。。。
	
	
	
	\textbf{In Question 3},
	%(第4段)	对于问题3我们用。。。。。。。。(模型的建立与求解结果的陈述中,思想、模型、软件和结果必须描述清晰,亮点详细说明需突出。针对不同问题可独立成段也可采用一段式仅用分号“;”分割,摘要只接受文字描述形式,不接受图表等其他方式)
	
	
	
	
	
	%(第5段)	优化结果及总结:在。。。。。。条件下,针对。。。。。。模型进行适当修改与优化,这种条件的改变可能来自你的一种猜想或建议。要注意合理性。此推广模型可以不深入研究,也可以没有具体结果。
	
	%注:字数300~600之间,需控制在一页;摘要中必须将具体方法、模型和所得结果写出来;摘要要求“总分总”,段开头可用“针对问题1,针对问题2,针对问题3..”或者“首先,然后,其次,最后”等词语进行有逻辑的论述。摘要是重中之重,必须严格执行!
	
	
	% 美赛论文中无需注明关键字。若您一定要使用,
	% 请将以下两行的注释号 '%' 去除,以使其生效
	\vspace{5pt}  %mm	毫米	1 mm = 2.845 pt   pt 点	1 pt = 0.351 mm
	\noindent\textbf{Keywords}: keyword 1,  keyword 2,  keyword 3,  keyword 4,  keyword 5
	%\keywords:关键词;\quad:空格
	%使用到的模型名称、方法名称、特别是亮点一定要在关键字里出现,3~5个较合适,用分号隔开
\end{abstract}
\maketitle  % 生成 Summary Sheet
%摘要  abstract.tex
	
	\tableofcontents  % 生成目录
	
	\input{texfile/1ProblemRestatement}%插入问题重述   ProblemRestatement.tex
	
	\newpage
\section{Problem Analysis}
%TOPS:这部分写差不多一页(删掉此句话)
\subsection{Analysis of Question One}
For question one,



\subsection{Analysis of Question Two} 
For question two,



\subsection{Analysis of Question Three}
For question three,



\subsection{Analysis of Question Four}
For question four,


%插入问题分析    ProblemAnalysis.tex
	
	\newpage%\新一页
\section{Assumptions and Justifications}
In response to the title of this article, the following hypotheses are proposed:
%1.	假设题目所给的数据真实可靠;
%注意:假设对整篇文章具有指导性,有时决定问题的难易。一定要注意假设的某种角度上的合理性,不能乱编,完全偏离事实或与题目要求相抵触。注意罗列要工整。

\begin{enumerate}[
	label = (\arabic*),
	itemindent = 0pt,
	labelindent = \parindent,
	labelwidth = 2em,
	labelsep = 5pt,
	leftmargin = 8em]
	\item[\textbf{Assumption 1:}] This is the first assumption 1.
	\item[\textbf{Assumption 2:}] This is the second assumption 2.
	\item[\textbf{Assumption 3:}] 
	\item[\textbf{Assumption 4:}] 
	\item[\textbf{Assumption 5:}] 
\end{enumerate}


\begin{enumerate}
	\item 
\end{enumerate}



\section{Notations}
% 这部分不要过页(删掉此句话)
\begin{center}
	\begin{tabular}{cccc}
		\toprule[1pt] 
		\makebox[0.15\textwidth][c]{Symbol} & \makebox[0.3\textwidth][c]{Description} & \makebox[0.15\textwidth][c]{Symbol} & \makebox[0.3\textwidth][c]{Description} \\  
		\hline
		$T_i$ & Temperature of small zone & $T_j$ & Temperature of large zone \\
		$x$ & Variable x & $y$ & Variable y \\    
		\bottomrule[1pt]
	\end{tabular}
\end{center}
\vspace{-0.5em}
\noindent{\small Note: Undefined variables are defined where they first appear.}


%插入模型假设及符号说明  AssumptionAndNotations.tex
	
	\newpage
\section{Data Preprocessing}


%1. 。。。。。。数据全部缺失,不予考虑。
%2. 对数据测试的特点,如周期等进行分析。
%3. 。。。。。。数据残缺,根据数据挖掘等理论根据。。。。。变化趋势进行补充。
%4. 对数据特点(后面将会用到的特征)进行提取。
%用。。。。。。。软件聚类分析和各个不同问题的需要,采得。。。组采样,每组5-8个采样值。将采样所对应的特征值进行列表或图示。
%根据数据特点,对总体和个体的特点进行比较,以表格或图示方式显示。
\begin{enumerate}[
	label = (\arabic*),
	itemindent = 0pt,
	labelindent = \parindent,
	labelwidth = 2em,
	labelsep = 5pt,
	leftmargin = 4.5em
	]
	\item[$Step$ 1:] good morning...
	\item[$Step$ 2:] good morning....
\end{enumerate}

\begin{enumerate}
	\item 
\end{enumerate}
 %插入数据预处理   DataPreprocessing.tex
	
	\newpage	
\section{Model Building and Solution}

\subsection{Model Establishment and Solution of Question One}
\subsubsection{***Model Establishment}


%模型建立的内容要点如下:
%模型的主要类别:
%几种常见的建模目的:
%建模过程常见的几个要点:
%模型的基本要求:
%模型选择要点:
%加分项(能在规定时间内做完后还有足够时间的再考虑加分项):
%1、鼓励创新。在能解决问题的基础上,对经典模型进行改进,欣赏独树一帜、有创新性的模型,但要合理。
%2、对于同一问题使用两个或以上合理模型进行求解。避免出现单纯罗列模型,又不做对比和评价的现象。
%参考话术:我们需要解决的问题是。。。。,题目要求是。。。。,剔除。。。数据后选用何种类型的模型优点进行分析。具体步骤123。。。
\subsubsection{***Model Solution}


%将预处理数据带入上述模型,通过。。。软件得到。。结果。(编程代码详见附件*)。模型求解及结果需要图文并茂,用数据说话  用图展示。具体步骤123。。。
\subsubsection{Results}


%针对于每一个问题的结果综述总结。
\subsection{Second-level Heading}
\subsubsection{Third-level Heading}


\subsubsection{List Environment}
\begin{enumerate}
	\item 
	
	\item 	
\end{enumerate}


\begin{itemize}
	\item[(a)] 
	
	
	\item[(b)] 	
\end{itemize}


\begin{itemize}
	\item 
	\item 
\end{itemize}

\subsubsection{Figure}	
\begin{figure}[h]%[h]:fixed action
	\centering%center
	\includegraphics [scale=0.5]{texfile/figures/1.png}
	\caption{Figure Name} 
	\label{fig:1}
\end{figure}

TOPS: How to cite, look here: \cref{fig:1}





\subsubsection{Table}

\begin{table}[!htbp]
	\caption{Table Name}	\centering
	\begin{tabular}{c c}
		\hline \multicolumn{1}{c} { Welding Zone Center Temperature} & Time $(\mathrm{s})$ \\
		\hline $30^{\circ} \mathrm{C}$ & 0 \\
		$150^{\circ} \mathrm{C}$ & $t_{1}$ \\
		$190^{\circ} \mathrm{C}$ & $t_{2}$ \\
		\hline
		\label{tab:1}
	\end{tabular}
\end{table}

TOPS: How to cite, look here: \cref{tab:1}

\subsubsection{Formula}

$\alpha^2+\beta^2=\gamma^2$

$$ \alpha^2+\beta^2=\gamma^2$$

\begin{equation}\nonumber
	\alpha^2+\beta^2=\gamma^2
\end{equation}

\begin{equation}
	\alpha^2+\beta^2=\gamma^2 \label{1}
\end{equation}

\begin{equation}
	a+b=c \label{2}
\end{equation}

TOPS: How to cite, look here: \cref{1,2}


How to  cite references, look here: \cite{ref01}, \cite{ref02} %插入模型的建立与求解   ModelBuilding.tex
	
	\newpage
\section{Sensitivity Analysis and Error Analysis}%有时间再做(没时间删去,影响不大)

\subsection{Sensitivity Analysis}
%灵敏度分析用于评估模型输入参数变化对输出结果的影响程度,识别关键参数


\subsection{Error Analysis for Question 1}
%最终数值结果的正确性或合理性是第一位的;对数值结果或模拟结果进行必要的检验。结果不正确、不合理、或误差大时,分析原因,对算法、计算方法、或模型进行修正、改进;





\subsection{Error Analysis for Question 2}
 %插入灵敏度分析和误差分析   SensitivityAndErrorAnalysis.tex
	
	\input{texfile/7ModelEvaluation} %插入模型评价	ModelEvaluation.tex
	
	\newpage
%参考文献
\clearpage
\label{MyLastPage}

% 自定义参考文献编号格式:括号对齐,编号居中
\makeatletter
\renewcommand\@biblabel[1]{[\makebox[1em][c]{#1}]}
\makeatother

\begin{thebibliography}{99}%宽度99,支持两位数编号对齐
	
	\setlength{\itemsep}{3mm}



\bibitem{ref01}Forsburg, Susan L. "Eukaryotic MCM proteins: beyond replication initiation." Microbiology and molecular biology reviews 68.1 (2004): 109-131.


%\cite{ref01}
\bibitem{ref02}Jenkin, M. E., J. C. Young, and A. R. Rickard. "The MCM v3. 3.1 degradation scheme for isoprene." Atmospheric Chemistry and Physics 15.20 (2015): 11433-11459.


%\cite{ref02}
\bibitem{ref03}


%\cite{ref03}
\bibitem{ref04}


%\cite{ref04}
\bibitem{ref05}


%\cite{ref05}
\bibitem{ref06}


%\cite{ref06}
\bibitem{ref07}

%\cite{ref07}
\bibitem{ref08}


%\cite{ref08}
\bibitem{ref09}


%\cite{ref09}
\bibitem{ref10}


%\cite{ref10}
	
	

\end{thebibliography}
 %插入参考文献   Reference.tex
	
	% 2025年美赛O奖论文大部分都没提供附录,如需提供请取消下一行注释
	% \newpage
%附录
\appendix
%附录需重新起页,论文附录至少应包括参赛论文的所有源程序代码,如实际使用的软件名称、命令和编写的全部可运行的源程序(含EXCEL、SPSS等软件的交互命令);通常还应包括自主查阅使用的数据等资料。赛题中提供的数据不要放在附录。如果缺少必要的源程序或程序不能运行(或者运行结果与正文不符),可能会被取消评奖资格。如果确实没有源程序,也应在论文附录中明确说明“本论文没有源程序”。 	

% 设置附录部分的编号格式和目录深度
\renewcommand{\thesubsection}{\Alph{subsection}}
\setcounter{tocdepth}{0}  % 附录内只显示section级别在目录中

\section*{Appendix}
\addcontentsline{toc}{section}{Appendix}

% 考虑到美赛页数限制,不做额外的数据图表展示,若有需要可取消注释。
% \subsection{Detailed Figures and Tables} % 索引方式	 \ref{xxtb1}  \ref{xxtb2}  \ref{xxtb3}
% 				  % 详细数据图请参见附录  \ref{xxtb1}。
% \subsubsection{******Detailed Data Table}%采用不编号格式
% \label{xxtb1}

% \subsubsection{******Detailed Data Table}
% \label{xxtb2}

% \subsubsection{******Detailed Data Table}
% \label{xxtb3}


\subsection{Code Programs}
% 插入Python代码
\lstinputlisting[style=Python, caption={Python Example}]{code/code_example.py}

% 插入MATLAB代码
\lstinputlisting[style=MATLAB, caption={MATLAB Example}]{code/code_example.m}








% 美赛不需要提供支撑材料
% \section{Supporting Materials}

% \begin{enumerate}
% 	\item 	Supporting Material 1
	
% 	\item 	Supporting Material 2
	
% 	\item 	Supporting Material 3
% \end{enumerate}

% 恢复目录深度设置
\setcounter{tocdepth}{3} %插入附录   Appendix.tex
	
	% 标记正文最后一页,AI报告不计入总页数
	% 手动标注“正文最后一页”(在 AI 声明之前)
	% \clearpage
	% \label{MyLastPage}

	% 美赛要求使用AI工具需声明,如使用请取消下一行注释
	 \AImatter
\addcontentsline{toc}{section}{Report on Use of AI}

% Suppress listing captions within the AI report
\captionsetup[lstlisting]{labelformat=empty,labelsep=none}

\begin{ReportAiUse}{9}
\item
Bing AI\\
Query1: write a sort algorithm\\
Output: Sure, I can help you with that. Here is a simple implementation of the selection sort algorithm in Python:
\begin{lstlisting}[language=python]
def selection_sort(arr):
    for i in range(len(arr)):
        min_idx = i
        for j in range(i + 1, len(arr)):
            if arr[min_idx] > arr[j]:
                min_idx = j
        arr[i], arr[min_idx] = arr[min_idx], arr[i]
    return arr
\end{lstlisting}
This algorithm sorts an array by repeatedly finding the minimum element from the unsorted part of the array and putting it at the beginning. \href{https://www.geeksforgeeks.org/sorting-algorithms/}{The time complexity of this algorithm is O($n^2$)}.

There are many other sorting algorithms available such as bubble sort, insertion sort, merge sort, quick sort, heap sort, and more. Each of these algorithms has its own advantages and disadvantages. You can learn more about them on GeeksforGeeks and Programiz.

I hope this helps!

\item
OpenAl Enie (Nov 5, 2023 version, Ernie 4.0)\\
Query1: <insert the exact wording of any subsequent input into the Al tool>\\
Output: <insert the complete output from the second query>

\item
Github CoPilot (Feb 3, 2024 version)\\
Query1: <insert the exact wording you input into the Al tool>\\
Output: <insert the complete output fron the Al tool>

\item
Google Bard (Feb 2,2024 version)\\
Query: <insert the exact wording of your query>\\
Output: <irsert the complete output from the Al tool>

\item
Bing AI\\
Query: write a FFT algorithm\\
Output: Sure, here is a Python implementation of the Fast Fourier Transform (FFT) algorithm:
\begin{lstlisting}[language=python]
import numpy as np

def fft(x):
    n = len(x)
    if n == 1:
        return x
    even = fft(x[0::2])
    odd = fft(x[1::2])
    T = [np.exp(-2j * np.pi * k / n) * odd[k] for k in range(n // 2)]
    return [even[k] + T[k] for k in range(n // 2)] + [even[k] - T[k] for k in range(n // 2)]
\end{lstlisting}
The FFT is a widely used algorithm for computing the \textbf{Discrete Fourier Transform (DFT)} of a sequence. It reduces the number of computations needed from O(n$^2$) to O(n log n) 12. The basic idea of the FFT is to apply divide and conquer. We divide the coefficient vector of the polynomial into two vectors, recursively compute the DFT for each of them, and combine the results to compute the DFT of the complete polynomial 3.

I hope this helps!

\end{ReportAiUse} %插入AI工具使用声明   AIToolDeclaration.tex

\end{document}
